%%%%%%%%%%%%%%%%%%%%%%%%%%%%%%%%%%%%%%%%%%%%%%%%%%%%%%%%%%%%%%%%%%%%%%%%
%%%%%%%%%%%%%%%%%%%%%% Simple LaTeX CV Template %%%%%%%%%%%%%%%%%%%%%%%%
%%%%%%%%%%%%%%%%%%%%%%%%%%%%%%%%%%%%%%%%%%%%%%%%%%%%%%%%%%%%%%%%%%%%%%%%

%%%%%%%%%%%%%%%%%%%%%%%%%%%%%%%%%%%%%%%%%%%%%%%%%%%%%%%%%%%%%%%%%%%%%%%%
%% NOTE: If you find that it says                                     %%
%%                                                                    %%
%%                           1 of ??                                  %%
%%                                                                    %%
%% at the bottom of your first page, this means that the AUX file     %%
%% was not available when you ran LaTeX on this source. Simply RERUN  %%
%% LaTeX to get the ``??'' replaced with the number of the last page  %%
%% of the document. The AUX file will be generated on the first run   %%
%% of LaTeX and used on the second run to fill in all of the          %%
%% references.                                                        %%
%%%%%%%%%%%%%%%%%%%%%%%%%%%%%%%%%%%%%%%%%%%%%%%%%%%%%%%%%%%%%%%%%%%%%%%%

%%%%%%%%%%%%%%%%%%%%%%%%%%%% Document Setup %%%%%%%%%%%%%%%%%%%%%%%%%%%%

% Don't like 10pt? Try 11pt or 12pt
\documentclass[10pt]{article}

% The automated optical recognition software used to digitize resume
% information works best with fonts that do not have serifs. This
% command uses a sans serif font throughout. Uncomment both lines (or at
% least the second) to restore a Roman font (i.e., a font with serifs).
%\usepackage{times}
%\renewcommand{\familydefault}{\sfdefault}

% This is a helpful package that puts math inside length specifications
\usepackage{calc}
\usepackage{comment}

% Simpler bibsection for CV sections
% (thanks to natbib for inspiration)
\makeatletter
\newlength{\bibhang}
\setlength{\bibhang}{1em} %1em}
\newlength{\bibsep}
 {\@listi \global\bibsep\itemsep \global\advance\bibsep by\parsep}
\newenvironment{bibsection}%
        {\begin{enumerate}{}{%
%        {\begin{list}{}{%
       \setlength{\leftmargin}{\bibhang}%
       \setlength{\itemindent}{-\leftmargin}%
       \setlength{\itemsep}{\bibsep}%
       \setlength{\parsep}{\z@}%
        \setlength{\partopsep}{0pt}%
        \setlength{\topsep}{0pt}}}
        {\end{enumerate}\vspace{-.6\baselineskip}}
%        {\end{list}\vspace{-.6\baselineskip}}
\makeatother

% Layout: Puts the section titles on left side of page
\reversemarginpar

%
%         PAPER SIZE, PAGE NUMBER, AND DOCUMENT LAYOUT NOTES:
%
% The next \usepackage line changes the layout for CV style section
% headings as marginal notes. It also sets up the paper size as either
% letter or A4. By default, letter was used. If A4 paper is desired,
% comment out the letterpaper lines and uncomment the a4paper lines.
%
% As you can see, the margin widths and section title widths can be
% easily adjusted.
%
% ALSO: Notice that the includefoot option can be commented OUT in order
% to put the PAGE NUMBER *IN* the bottom margin. This will make the
% effective text area larger.
%
% IF YOU WISH TO REMOVE THE ``of LASTPAGE'' next to each page number,
% see the note about the +LP and -LP lines below. Comment out the +LP
% and uncomment the -LP.
%
% IF YOU WISH TO REMOVE PAGE NUMBERS, be sure that the includefoot line
% is uncommented and ALSO uncomment the \pagestyle{empty} a few lines
% below.
%

%% Use these lines for letter-sized paper
\usepackage[paper=letterpaper,
            %includefoot, % Uncomment to put page number above margin
            marginparwidth=1.2in,     % Length of section titles
            marginparsep=.05in,       % Space between titles and text
            margin=1in,               % 1 inch margins
            includemp]{geometry}

%% Use these lines for A4-sized paper
%\usepackage[paper=a4paper,
%            %includefoot, % Uncomment to put page number above margin
%            marginparwidth=30.5mm,    % Length of section titles
%            marginparsep=1.5mm,       % Space between titles and text
%            margin=25mm,              % 25mm margins
%            includemp]{geometry}

%% More layout: Get rid of indenting throughout entire document
\setlength{\parindent}{0in}

\usepackage[shortlabels]{enumitem}

%% Reference the last page in the page number
%
% NOTE: comment the +LP line and uncomment the -LP line to have page
%       numbers without the ``of ##'' last page reference)
%
% NOTE: uncomment the \pagestyle{empty} line to get rid of all page
%       numbers (make sure includefoot is commented out above)
%
\usepackage{fancyhdr,lastpage}
\pagestyle{fancy}
%\pagestyle{empty}      % Uncomment this to get rid of page numbers
\fancyhf{}\renewcommand{\headrulewidth}{0pt}
\fancyfootoffset{\marginparsep+\marginparwidth}
\newlength{\footpageshift}
\setlength{\footpageshift}
          {0.5\textwidth+0.5\marginparsep+0.5\marginparwidth-2in}
\lfoot{\hspace{\footpageshift}%
       \parbox{4in}{\, \hfill %
                    \arabic{page} of \protect\pageref*{LastPage} % +LP
%                    \arabic{page}                               % -LP
                    \hfill \,}}

% Finally, give us PDF bookmarks
\usepackage{color,hyperref}
\definecolor{darkblue}{rgb}{0.0,0.0,0.45}
\hypersetup{colorlinks,breaklinks,
            linkcolor=darkblue,urlcolor=darkblue,
            anchorcolor=darkblue,citecolor=darkblue}

%%%%%%%%%%%%%%%%%%%%%%%% End Document Setup %%%%%%%%%%%%%%%%%%%%%%%%%%%%


%%%%%%%%%%%%%%%%%%%%%%%%%%% Helper Commands %%%%%%%%%%%%%%%%%%%%%%%%%%%%

% The title (name) with a horizontal rule under it
% (optional argument typesets an object right-justified across from name
%  as well)
%
% Usage: \makeheading{name}
%        OR
%        \makeheading[right_object]{name}
%
% Place at top of document. It should be the first thing.
% If ``right_object'' is provided in the square-braced optional
% argument, it will be right justified on the same line as ``name'' at
% the top of the CV. For example:
%
%       \makeheading[\emph{Curriculum vitae}]{Your Name}
%
% will put an emphasized ``Curriculum vitae'' at the top of the document
% as a title. Likewise, a picture could be included:
%
%   \makeheading[\includegraphics[height=1.5in]{my_picutre}]{Your Name}
%
% the picture will be flush right across from the name.
\newcommand{\makeheading}[2][]%
        {\hspace*{-\marginparsep minus \marginparwidth}%
         \begin{minipage}[t]{\textwidth+\marginparwidth+\marginparsep}%
             {\large \bfseries #2 \hfill #1}\\[-0.15\baselineskip]%
                 \rule{\columnwidth}{1pt}%
         \end{minipage}}

% The section headings
%
% Usage: \section{section name}
\renewcommand{\section}[1]{\pagebreak[3]%
    \hyphenpenalty=10000%
    \vspace{1.3\baselineskip}%
    \phantomsection\addcontentsline{toc}{section}{#1}%
    \noindent\llap{\scshape\smash{\parbox[t]{\marginparwidth}{\raggedright #1}}}%
    \vspace{-\baselineskip}\par}

% An itemize-style list with lots of space between items
\newenvironment{outerlist}[1][\enskip\textbullet]%
        {\begin{itemize}[#1,leftmargin=*]}{\end{itemize}%
         \vspace{-.6\baselineskip}}

% An environment IDENTICAL to outerlist that has better pre-list spacing
% when used as the first thing in a \section
\newenvironment{lonelist}[1][\enskip\textbullet]%
        {\begin{list}{#1}{%
        \setlength{\partopsep}{0pt}%
        \setlength{\topsep}{0pt}}}
        {\end{list}\vspace{-.6\baselineskip}}

% An itemize-style list with little space between items
\newenvironment{innerlist}[1][\enskip\textbullet]%
        {\begin{itemize}[#1,leftmargin=*,parsep=0pt,itemsep=0pt,topsep=0pt,partopsep=0pt]}
        {\end{itemize}}

% An environment IDENTICAL to innerlist that has better pre-list spacing
% when used as the first thing in a \section
\newenvironment{loneinnerlist}[1][\enskip\textbullet]%
        {\begin{itemize}[#1,leftmargin=*,parsep=0pt,itemsep=0pt,topsep=0pt,partopsep=0pt]}
        {\end{itemize}\vspace{-.6\baselineskip}}

% To add some paragraph space between lines.
% This also tells LaTeX to preferably break a page on one of these gaps
% if there is a needed pagebreak nearby.
\newcommand{\blankline}{\quad\pagebreak[3]}
\newcommand{\halfblankline}{\quad\vspace{-0.5\baselineskip}\pagebreak[3]}

% Uses hyperref to link DOI
\newcommand\doilink[1]{\href{http://dx.doi.org/#1}{#1}}
\newcommand\doi[1]{doi:\doilink{#1}}

% For \url{SOME_URL}, links SOME_URL to the url SOME_URL
\providecommand*\url[1]{\href{#1}{#1}}
% Same as above, but pretty-prints SOME_URL in teletype fixed-width font
\renewcommand*\url[1]{\href{#1}{\texttt{#1}}}

% For \email{ADDRESS}, links ADDRESS to the url mailto:ADDRESS
\providecommand*\email[1]{\href{mailto:#1}{#1}}
% Same as above, but pretty-prints ADDRESS in teletype fixed-width font
%\renewcommand*\email[1]{\href{mailto:#1}{\texttt{#1}}}

%\providecommand\BibTeX{{\rm B\kern-.05em{\sc i\kern-.025em b}\kern-.08em
%    T\kern-.1667em\lower.7ex\hbox{E}\kern-.125emX}}
%\providecommand\BibTeX{{\rm B\kern-.05em{\sc i\kern-.025em b}\kern-.08em
%    \TeX}}
\providecommand\BibTeX{{B\kern-.05em{\sc i\kern-.025em b}\kern-.08em
    \TeX}}
\providecommand\Matlab{\textsc{Matlab}}

%%%%%%%%%%%%%%%%%%%%%%%% End Helper Commands %%%%%%%%%%%%%%%%%%%%%%%%%%%

%%%%%%%%%%%%%%%%%%%%%%%%% Begin CV Document %%%%%%%%%%%%%%%%%%%%%%%%%%%%

\begin{document}
\makeheading{{\Large Curriculum Vitae}\vspace{2 mm} \\Mahdi Ebrahimi Kahou \hfill {\tiny Last Update: October 2022}} \\
\vspace{5 mm}

\section{Personal Information}\vspace{3 mm}
\makeheading{}\vspace{2 mm}
% NOTE: Mind where the & separators and \\ breaks are in the following
%       table.
%
% ALSO: \rcollength is the width of the right column of the table
%       (adjust it to your liking; default is 1.85in).
%
\newlength{\rcollength}\setlength{\rcollength}{1.4in}%
%
\begin{tabular}[t]{@{}p{\textwidth-\rcollength}p{\rcollength}}

  
 Citizenship:  Canadian, Iranian  \\ 
 Contact Info: \href{mekahou@alumni.ubc.ca}{\email{mekahou@alumni.ubc.ca}}, \href{https://sites.google.com/site/mahdiebrahimikahou/} {Website}.
                      
\end{tabular}

%\section{Objective}

%Insert text here if you want to
%\begin{innerlist}
%\item More information and auxiliary documents can be found at\\\url{http://www.tedpavlic.com/facjobsearch/}
%\end{innerlist}

\section{Research Interests}\vspace{3 mm}
\makeheading{}\vspace{2 mm}
Macroeconomics, Machine Learning, Econometrics, and Computational Economics.

\section{Education}\vspace{0 mm}
\makeheading{}\vspace{2 mm}
\begin{outerlist}

\item Doctor of Philosophy in Economics, \href{https://economics.ubc.ca/}
              {University British Columbia}. \hfill {2017-}\\
              Committee:
              \href{https://www.jesseperla.com/}
              {Dr. Jesse Perla} (UBC), \href{https://www.sas.upenn.edu/~jesusfv/}
              {Dr. Jesús Fernández-Villaverde} (UPenn),\\ \href{https://hkasahar.arts.ubc.ca/}
              {Dr. Hiroyuki Kasahara} (UBC),  \href{https://economics.ubc.ca/faculty-and-staff/vadim-marmer/}
              { Dr. Vadim Marmer} (UBC)
              \begin{outerlist}
              	\item Visiting scholar at Federal Reserve Bank of Minneapolis. \hfill October 2022
              	\vspace{1mm} 
              \end{outerlist}
\end{outerlist}

\begin{outerlist}
\item Doctor of Philosophy in Economics, \href{http://cla.umn.edu/economics}
              {University of Minnesota}.  \\
             Voluntary withdrawal due to the U.S. travel ban after finishing the first year.
%\item Doctor of Philosophy in Finance, \href{http://www.haskayne.ucalgary.ca/}
%             {Haskayne School of Business}, \hspace{2mm}  \hfill {2013-2016}\\
%             University of Calgary,
%Calgary, AB.
%            
%       
%       Supervisor:
%             \href{http://homepages.ucalgary.ca/~alehar/}
%                   {Dr. Alfred Lehar}\\
%                    %GPA: 3.98   
%                       
             
\end{outerlist}
\begin{outerlist}
\item Master of Science in Physics, \hspace{2mm} \\ \href{http://www.iqst.ca/}{Institute for Quantum Science and Technology}, \href{http://www.ucalgary.ca/}{University of Calgary}.
           
         Thesis Topic: \href{http://theses.ucalgary.ca/handle/11023/542}
             {Spatial search via non-linear quantum walk}\\
         Supervisor:
              \href{http://www.iqst.ca/people/home/dfeder/index.php}
                  {Dr. David Feder}
         %GPA: 4.0      
       
\end{outerlist}


\begin{outerlist}
\item Bachelor of Science in Physics, 
             \href{http://www.sharif.ir/web/en/}{Sharif University of Technology}.
      % GPA: 17.71/20 (3.9)\\
       %Rank 3rd out of 50, Physics Department, \\ Sharif
%University of Technology\\
       
\end{outerlist}

\section{Working Papers}\vspace{0 mm}
\makeheading{}\vspace{2 mm}
\begin{innerlist}

\item {\bf Job market paper:} \href{https://www.nber.org/papers/w28981}{``Exploiting Symmetry in High-Dimensional Dynamic Programming"}, NBER Working Paper, 2021.   \\
With Jesús Fernández-Villaverde, Jesse Perla, and Arnav Sood.	\\
``We propose a new method for solving high-dimensional dynamic programming problems and recursive competitive equilibria with a large (but finite) number of heterogeneous agents using deep learning. We avoid the curse of dimensionality thanks to three complementary techniques: (1) exploiting symmetry in the  law of motion and the value function; (2) constructing a concentration of measure to calculate high-dimensional expectations using a single Monte Carlo draw from the distribution of idiosyncratic shocks; and (3) designing and training deep learning architectures that exploit symmetry and concentration of measure. As an application, we find a global solution of a multi-firm version of the classic Lucas and Prescott (1971) model of investment under uncertainty. First, we compare the solution against a linear-quadratic Gaussian version for validation and benchmarking. Next, we solve the nonlinear version where no accurate or closed-form solution exists. Finally, we describe how our approach applies to a large class of models in economics. "\\

	
\item \href{https://mekahou.github.io/docs/Papers/SpookyBoundary.pdf}{``Spooky Boundaries at a Distance: Exploring Transversality and Stability with Deep Learning".} \\
With Jesús Fernández-Villaverde, Sebastian Gomez Cardona, Jesse Perla, and Jan Rosa.\\
``In the long run, we are all dead. Nonetheless, even when investigating short-run dynamics, models require boundary conditions on long-run, forward-looking behavior (e.g., transversality and no-bubble conditions). In this paper, we show how deep learning approximations can automatically fulfill these conditions despite not directly calculating the steady state, balanced growth path, or ergodic distribution. The main implication is that we can solve for transition dynamics with forward-looking agents, confident that long-run boundary conditions will implicitly discipline the short-run decisions, even converging towards the correct equilibria in cases with steady-state multiplicity. While this paper analyzes benchmarks such as the neoclassical growth model, the results suggest deep learning may let us calculate accurate transition dynamics with high-dimensional state spaces, and without directly solving for long-run behavior."\\

\item \href{https://mekahou.github.io/docs/Papers/input_output.pdf}{``Optimal Entry Decision with Correlated Variable Cost and Output Price"}. \\
``In models with irrecoverable investment and uncertainty in the output price it is a well-established result that uncertainty increases the output price that a firm starts investment. This paper studies a model of irrecoverable investment (entry) where the variable cost and output price are characterized by two correlated geometric Brownian motions. The numerical results indicate that in the presence of high levels of correlation the impact of uncertainty in output price is ambiguous and depends on the level of variable cost. Specifically, increasing uncertainty in output prices increases the entry output price for low levels of variable cost and the reverse happens for high levels of variable cost. Therefore, in the presence of high levels of correlation the conventional result does not hold anymore. Moreover, this study establishes that increasing the correlation level decreases the entry output price." 
\end{innerlist}

\section{Work in progress}\vspace{3 mm}
\makeheading{}\vspace{2 mm}
\begin{innerlist}
\item ``Solving Equilibrium Economic Models with Deep Learning". \\
With Jesús Fernández-Villaverde and Jesse Perla.\\
``The success of deep learning in a variety of applications is leading economists to explore its potential for orders of magnitude increase in the size of the state-space and the complexity of models we can solve. In this paper, we provide a clear mental picture of what deep learning is, how it relates to existing solution methods, how to encode economic insights and domain knowledge, and where the methods are likely to be revolutionary. In answering those questions, we demystify these methods, explain the core concepts with simple, however, insightful examples, and debunk some “folk wisdom” commonly held, while elaborating on places where we should proceed with caution."
\end{innerlist}
\section{Publications}\vspace{0 mm}
\makeheading{}\vspace{2 mm}
\begin{innerlist}
  \item \href{http://pra.aps.org/abstract/PRA/v88/i3/e032310}{``Quantum Search with Interacting Bose-Einstein Condensates"},  Physical Review A,   2013. (With  David L. Feder)
  
  \item\href{http://www.sciencedirect.com/science/article/pii/S1572308916302297}{{ ``Macroprudential Policy:~A Review"}}, Journal of Financial Stability, 2017. (With Alfred Lehar)
\end{innerlist}

\newpage
\section{Scholarships, Awards and Honours}\vspace{7 mm}
\makeheading{}\vspace{2 mm}
\begin{innerlist}
\item Selected for 71st Lindau Nobel Laureate Meeting on Economic Sciences. 
\item Social Sciences and Humanities Research Council (SSHRC) Doctoral Fellowship.
\item Four Year Doctoral Fellowship (UBC).
%\item Eyes High International Doctoral Scholarship ($\times 2$) %\hfill 2015-2016 %($\$ 12,000$)  \hfill 2015-2016
%\item Dr. Murray Fraser Memorial Graduate Scholarship %\hfill 2015%($\$ 800$) 
%\item Eyes High International Doctoral Scholarship %\hfill 2014-2015%($\$ 12,000$)  
%\item Marion Janet \& Ian Stormont Forbes Scholarship %\hfill 2014-2015%($\$ 8,000$) 
%\item AIMCO Graduate Scholarship in Finance %\hfill 2014%($\$ 10,000$) 
\item Faculty of Graduate Studies Award (University of Calgary). %\hfill  2012%($\$ 2,500$)  
%\item Full tuition fee scholarship [Competitive] %\hfill 2010-2012%($\$ 9,000$) 
\item Institute for Quantum Science and Technology ``Top-Up Students Award"(University of Calgary). %\hfill 2011%($\$ 2,000$)  
\item Bronze medal in National Physics Olympiad. %\hfill 2004 %
\end{innerlist}



\section{Presentations}\vspace{3 mm}
\makeheading{}\vspace{2 mm}
\begin{innerlist}
\item 2023: University of Toronto (Scheduled).
\item 2022:  Society For Economic Dynamics (Madison), University of Minnesota, University of Colorado Boulder.
\item 2012: Southwest Quantum Information and Technology(Albuquerque), Canadian Association of Physicists (Calgary).
\end{innerlist}



%\section{Other Publications}\vspace{3 mm}
%
%\makeheading{}\vspace{2 mm}
%\begin{innerlist}
% 
%    \item Mahdi Ebrahimi Kahou ,\href{http://theses.ucalgary.ca/handle/11023/542}{``Spatial search via non-linear quantum walk."}  \emph{M.Sc. Thesis}, University of Calgary Current Theses Repository - The Vault, January 2013 .\\
  
%\end{innerlist}

%\section{Languages}\vspace{0 mm}
%\makeheading{}\vspace{2 mm}
%\begin{innerlist}
%\item English
%\item Farsi
%\end{innerlist}

\section{References}\vspace{0 mm}
\makeheading{}\vspace{2 mm}
\begin{innerlist}
\item Dr. Jesse Perla \\ 
Associate Professor of Economics \hfill{Phone: +1 604 822-5721}  \\ 
University of British Columbia \hfill{E-mail: \email{jesse.perla@ubc.ca}}.\\
6000 Iona Drive, Vancouver, BC Canada V6T 1L4. 

\end{innerlist}
\halfblankline
\begin{innerlist}
	\item Dr.  Jesús Fernández-Villaverde\\
	Professor of Economics \hfill{Phone: +1 215 573-1504} \\ 
    University of Pennsylvania 	\hfill{E-mail: \email{jesusfv@econ.upenn.edu}}\\
    The Ronald O. Perelman Center for Political Science and Economics\\
    133 South 36th Street, Philadelphia, PA 19104. 
\end{innerlist}
\halfblankline
\begin{innerlist}
	\item Dr. Hiroyuki Kasahara\\
     Professor of Economics \hfill{Phone: +1 604 822-4814}\\ 
	University of British Columbia \hfill{E-mail: \email{Hiroyuki.Kasahara@ubc.ca}}\\
	6000 Iona Drive, Vancouver, BC Canada V6T 1L4.
\end{innerlist}
\halfblankline
\begin{innerlist}
	\item Dr. Vadim Marmer\\
	Professor of Economics   \hfill{Phone: +1 604 822-8217}\\ 
	University of British Columbia \hfill{E-mail: \email{Vadim.Marmer@ubc.ca}}\\
	6000 Iona Drive, Vancouver, BC Canada V6T 1L4 .
\end{innerlist}

\section{Teaching Experience}\vspace{3 mm}
\makeheading{}\vspace{2 mm}
\begin{innerlist}
	\item TA for Computational Economics with Data Science Applications, UBC. \hfill 2019 
	\item TA for Information and Incentive (graduate), UBC.\hfill 2019
	\item TA for PhD Econometrics Theory, UBC.\hfill 2018
	\item TA for honor level macroeconomics and microeconomics, UBC. \hfill 2019-2020 
	\item TA Economic Growth, University of Minnesota. \hfill 2016
	\item TA for Phys 211/221, Phys 259 and Phys 369, University of Calgary.  \hfill 2010-2012
	\item TA for Analytical Mechanics I\&II, Sharif University of Technology. \hfill 2007-2008
	%\item Physics Olympiad teacher in several branches of NODET (National
	%Organization for Development of Exceptional Talents). \hfill 2004-2010
\end{innerlist}


\section{Computer Programming}\vspace{3 mm}
\makeheading{}\vspace{2 mm}

\begin{innerlist}
	\item \Matlab, Mathematica, Julia, Python.  
\end{innerlist}


\end{document}

